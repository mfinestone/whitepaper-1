\documentclass[9pt,oneside]{amsart}
%\usepackage{tweaklist}
\usepackage{cancel}
\usepackage{xspace}
\usepackage{graphicx}
\usepackage{multicol}
\usepackage{subfig}
\usepackage{amsmath}
\usepackage{amssymb}
\usepackage[a4paper,width=170mm,top=18mm,bottom=22mm,includeheadfoot]{geometry}
\usepackage{booktabs}
\usepackage{array}
\usepackage{verbatim}
\usepackage{caption}
\usepackage{float}
\usepackage{pdflscape}
\usepackage{mathtools}
\usepackage[usenames,dvipsnames]{xcolor}
\usepackage{afterpage}
\usepackage{tikz}
\usepackage[bookmarks=true, unicode=true, pdftitle={Taiko: A Type-1 ETHEREUM ZK-ROLLUP}, pdfauthor={Taiko Labs},pdfkeywords={Ethereum, White Paper, blockchain,EVM, zkEVM, ZK-Rollup, Layer 2},pdfborder={0 0 0.5 [1 3]}]{hyperref}
%,pagebackref=true
% \usepackage{easy-todo}
\usepackage{todonotes}
\newcommand{\todoMev}{\todo[color=green!40, inline]}

\usepackage{tabu} %requires array.

\PassOptionsToPackage{hyphens}{url}\usepackage{hyperref}


\makeatletter
 \newcommand{\linkdest}[1]{\Hy@raisedlink{\hypertarget{#1}{}}}
\makeatother
\usepackage{seqsplit}

\usepackage{soul}
\usepackage[english]{babel}
\usepackage[autostyle]{csquotes}
\MakeOuterQuote{"}

\usepackage[final]{microtype} % 
% Default rendering options
\definecolor{pagecolor}{rgb}{1,1,1}
\setul{0.1ex}{0.05ex}
\def\VersionNumber{0.1.0}
\IfFileExists{Options.tex}{\input{Options.tex}}

\newcommand{\hcancel}[1]{%
    \tikz[baseline=(tocancel.base)]{
        \node[inner sep=0pt,outer sep=0pt] (tocancel) {#1};
        \draw[black] (tocancel.south west) -- (tocancel.north east);
    }%
}%


\DeclarePairedDelimiter{\ceil}{\lceil}{\rceil}
\newcommand*\eg{e.g.\@\xspace}
\newcommand*\Eg{e.g.\@\xspace}
\newcommand*\ie{i.e.\@\xspace}
\newcommand{\thedate}{\today}
%\renewcommand{\itemhook}{\setlength{\topsep}{0pt}  \setlength{\itemsep}{0pt}\setlength{\leftmargin}{15pt}}

\begin{document}
\setlength{\columnsep}{20pt}
\begin{multicols}{2}


\section{Taiko Block Fees}
We assume EIP-1559 will be enabled on L2 to adjust L2 block gas price and gas limit.

Let $K_{\mathrm{MaxProposedBlocks}}$ be the max number of slots for proposed blocks; $m$ be the current number of available slots, ; $p$ be the fee for proposing the first L2 block. we maintain two internal variables $m$ and $k$, whose values are initialized to $K_{\mathrm{MaxProposedBlocks}}$ and $pK_{\mathrm{MaxProposedBlocks}}$, respectively.

We adopt the idea from \textit{Automatic Market Making}, and require that when $m$ changes to a new value $m'$, $k$ must also change to a new value $k'$ such that:

Then we update k as follows:
\begin{eqnarray}
k'(m'+1) \equiv k(m+1)
\end{eqnarray}


\subsection{Proposing Fees}
If $n$ blocks are proposed in one batch, then we have:

\begin{eqnarray}
k'(m-n+1)  \equiv   k(m+1)
\end{eqnarray}

thus:

\begin{eqnarray}
k'  \equiv \frac{k(m+1)}{m-n+1}
\end{eqnarray}

The difference between $k$ and $k'$ is the total proposing fee for these $n$ blocks:

\begin{eqnarray}
f(n)  \equiv  k' - k  \equiv   \frac{kn}{m-n+1}  
\end{eqnarray}

therefore, the average proposal fee per block in this batch is 

\begin{eqnarray}
\bar{f}(n)   \equiv \frac{f(n)}{n} \equiv \frac{k}{m-n+1}
\end{eqnarray}


Usually, blocks are proposed one after another, therefore, the block proposing fee is:

\begin{eqnarray}
f(1)  \equiv \bar{f}(1)  \equiv \frac{k}{m}
\end{eqnarray}



The proposal fee for the very first block is:



\begin{eqnarray}
f_1  \equiv \frac{k}{m}  \equiv \frac{pK_{\mathrm{MaxProposedBlocks}}}{K_{\mathrm{MaxProposedBlocks}}} \equiv p
\end{eqnarray}


\subsection{Proving Fees}

Provers are paid with a proving fee when their blocks are finalized. Suppose $n$ blocks are finalized in one batch, then we have:

\begin{eqnarray}
k'  (m+n+1) & \equiv  & k(m+1) \\
\nonumber k'  & \equiv & \frac{k(m+1)}{m+n+1} 
\end{eqnarray}



The total proving fee is:

\begin{eqnarray}
f(n)  \equiv  k - k'  \equiv \frac{kn}{m+n+1} 
\end{eqnarray}


Each block's prover receive proving fee:


\begin{eqnarray}
\bar{f}(n)  \equiv \frac{f(n)}{n}  \equiv \frac{k}{m+n+1} 
\end{eqnarray}

\subsection{Adjusting fees based on proving time}
when a block is proven, we update two proving delay moving average:  $a^1$ is a longer term-moving average that changes more slowly, formally
$a^1_{i+1} \equiv (a^1_{i} * 1023 + d_i)/1024$;  $a^2$ is a short term-moving average that changes more rapidly, formally
$a^2_{i+1} \equiv (a^2_{i} * 63 + d_i)/64$.

Then we update k as follows:
\begin{eqnarray}
\mu & \equiv & \min(\max( \frac{a^2}{a^1}, \frac{1}{0.95}), 0.95) \\
k' & \equiv  & \mu k  
\end{eqnarray}

$\mu$ is called the \textit{fee market adjustment factor}.

\subsection{TKO as Protocol-Level Fee Token}
I strongly recommend to use TKO, Taiko's governance token,  as the block proposing and proving fee token. When a block is proposed, TKO is burned from the proposer address; when a block is finalized, TKO is minted to the prover address, which features a dynamic total token supply determined by the ZKP proving market.

For a tokenomics to work for the long run, there must be both demands and supplies. This model offers such a property.
\end{multicols}

\end{document}